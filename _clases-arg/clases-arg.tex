% Options for packages loaded elsewhere
\PassOptionsToPackage{unicode}{hyperref}
\PassOptionsToPackage{hyphens}{url}
%
\documentclass[
]{article}
\usepackage{amsmath,amssymb}
\usepackage{lmodern}
\usepackage{iftex}
\ifPDFTeX
  \usepackage[T1]{fontenc}
  \usepackage[utf8]{inputenc}
  \usepackage{textcomp} % provide euro and other symbols
\else % if luatex or xetex
  \usepackage{unicode-math}
  \defaultfontfeatures{Scale=MatchLowercase}
  \defaultfontfeatures[\rmfamily]{Ligatures=TeX,Scale=1}
\fi
% Use upquote if available, for straight quotes in verbatim environments
\IfFileExists{upquote.sty}{\usepackage{upquote}}{}
\IfFileExists{microtype.sty}{% use microtype if available
  \usepackage[]{microtype}
  \UseMicrotypeSet[protrusion]{basicmath} % disable protrusion for tt fonts
}{}
\makeatletter
\@ifundefined{KOMAClassName}{% if non-KOMA class
  \IfFileExists{parskip.sty}{%
    \usepackage{parskip}
  }{% else
    \setlength{\parindent}{0pt}
    \setlength{\parskip}{6pt plus 2pt minus 1pt}}
}{% if KOMA class
  \KOMAoptions{parskip=half}}
\makeatother
\usepackage{xcolor}
\IfFileExists{xurl.sty}{\usepackage{xurl}}{} % add URL line breaks if available
\IfFileExists{bookmark.sty}{\usepackage{bookmark}}{\usepackage{hyperref}}
\hypersetup{
  pdftitle={Libro de Cocina para el Análisis de las Clases Sociales en Argentina},
  pdfauthor={Nicolás Sacco; José Rodríguez de la Fuente; Sofia Jaime},
  hidelinks,
  pdfcreator={LaTeX via pandoc}}
\urlstyle{same} % disable monospaced font for URLs
\usepackage[margin=1in]{geometry}
\usepackage{longtable,booktabs,array}
\usepackage{calc} % for calculating minipage widths
% Correct order of tables after \paragraph or \subparagraph
\usepackage{etoolbox}
\makeatletter
\patchcmd\longtable{\par}{\if@noskipsec\mbox{}\fi\par}{}{}
\makeatother
% Allow footnotes in longtable head/foot
\IfFileExists{footnotehyper.sty}{\usepackage{footnotehyper}}{\usepackage{footnote}}
\makesavenoteenv{longtable}
\usepackage{graphicx}
\makeatletter
\def\maxwidth{\ifdim\Gin@nat@width>\linewidth\linewidth\else\Gin@nat@width\fi}
\def\maxheight{\ifdim\Gin@nat@height>\textheight\textheight\else\Gin@nat@height\fi}
\makeatother
% Scale images if necessary, so that they will not overflow the page
% margins by default, and it is still possible to overwrite the defaults
% using explicit options in \includegraphics[width, height, ...]{}
\setkeys{Gin}{width=\maxwidth,height=\maxheight,keepaspectratio}
% Set default figure placement to htbp
\makeatletter
\def\fps@figure{htbp}
\makeatother
\setlength{\emergencystretch}{3em} % prevent overfull lines
\providecommand{\tightlist}{%
  \setlength{\itemsep}{0pt}\setlength{\parskip}{0pt}}
\setcounter{secnumdepth}{5}
\newlength{\cslhangindent}
\setlength{\cslhangindent}{1.5em}
\newlength{\csllabelwidth}
\setlength{\csllabelwidth}{3em}
\newlength{\cslentryspacingunit} % times entry-spacing
\setlength{\cslentryspacingunit}{\parskip}
\newenvironment{CSLReferences}[2] % #1 hanging-ident, #2 entry spacing
 {% don't indent paragraphs
  \setlength{\parindent}{0pt}
  % turn on hanging indent if param 1 is 1
  \ifodd #1
  \let\oldpar\par
  \def\par{\hangindent=\cslhangindent\oldpar}
  \fi
  % set entry spacing
  \setlength{\parskip}{#2\cslentryspacingunit}
 }%
 {}
\usepackage{calc}
\newcommand{\CSLBlock}[1]{#1\hfill\break}
\newcommand{\CSLLeftMargin}[1]{\parbox[t]{\csllabelwidth}{#1}}
\newcommand{\CSLRightInline}[1]{\parbox[t]{\linewidth - \csllabelwidth}{#1}\break}
\newcommand{\CSLIndent}[1]{\hspace{\cslhangindent}#1}
\usepackage{booktabs}
\usepackage{longtable}
\usepackage{array}
\usepackage{multirow}
\usepackage{wrapfig}
\usepackage{float}
\usepackage{colortbl}
\usepackage{pdflscape}
\usepackage{tabu}
\usepackage{threeparttable}
\usepackage{threeparttablex}
\usepackage[normalem]{ulem}
\usepackage{makecell}
\usepackage{xcolor}
\usepackage{multicol}
\usepackage{hhline}
\usepackage{hyperref}
\ifLuaTeX
  \usepackage{selnolig}  % disable illegal ligatures
\fi

\title{Libro de Cocina para el Análisis de las Clases Sociales en Argentina}
\author{Nicolás Sacco \and José Rodríguez de la Fuente \and Sofia Jaime}
\date{2022-03-30}

\begin{document}
\maketitle

{
\setcounter{tocdepth}{2}
\tableofcontents
}
\hypertarget{prefacio}{%
\section*{Prefacio}\label{prefacio}}
\addcontentsline{toc}{section}{Prefacio}

La literatura sobre las clases sociales en Argentina posee ya una larga tradición y una amplia gama de abordajes. La relevancia de este tema reside en las transformaciones recientes de la estructura social, pero también, en los desafíos, tanto teóricos como metodológicos, que el tema posee. Estudiantes, investigadores y profesionales, en fin, aquellos interesados en su estudio, se encuentran de forma frecuente con la paralizante tarea de afrontar la infinita literatura y discusión teórica sobre la cuestión, la construcción de información, o bien con el oscuro privilegio de acceso a ciertas bases de datos, en el caso de los estudios con datos cuantitativos secundarios; en definitiva, en la posibilidad de caer en las trampas de la ciencia cerrada o no-reproducible, todavía bastante frecuente.

A modo de aporte para cubrir parte de estos problemas, este manual, abierto a la comunidad para su consulta, ofrece a través de ``recetas'' prácticas situar a los lectores en la ``cocina'' de la investigación para el estudio de las clases sociales, en particular, en el abordaje de la problemática de la construcción y análisis estadístico de datos para su estudio en la Argentina contemporánea, con un enfoque global y de largo plazo, en base a datos cuantitativos secundarios.

Compartiendo información compilada y herramientas usualmente dispersas tanto en la literatura, como en bibliotecas o bases de datos de acceso exclusivo, este libro condensa algunas lecciones aprendidas y experiencias de investigación, apoyados en el lenguaje de programación \textbf{\texttt{R}} y con la interfaz \textbf{\texttt{RStudio}}. Siguiendo el criterio general de ciencia abierta y reproducible, \textbf{\texttt{R}} permite ejemplificar el procedimiento de gestión de bases de datos y de procesamientos estadísticos, ya que se trata de una poderosa herramienta para la estadística, presentación gráfica, y programación, utilizada por miles de usuarios. Pero a la vez, el uso de \textbf{\texttt{R}} puede ser desalentador. Por ello, en este libro ofrecemos soluciones de programación para problemas específicos.

Esta iniciativa se orientó, en ese sentido, al desarrollo de un conjunto básico de contenidos e instrumentos actualizados para la gestión de datos y análisis de las clases sociales. Este manual incluye su socialización en un formato también plano, y la presentación de herramientas computacionales que permiten apoyar la aplicación de los métodos presentados.

A partir del relevamiento, evaluación, ajuste y procesamiento, los lectores serán expuestos a los desafíos empíricos y metodológicos de encarar la construcción de datos en un campo de estudio donde no predomina información para largos períodos históricos que involucran a la población de hecho en Argentina. Utilizando principalmente como fuentes de datos a los censos de población modernos (1970-2010) pero, sobre todo, a la Encuesta Permanente de Hogares (1974-2020) del Instituto Nacional de Estadística y Censos (INDEC), en los capítulos que siguen se retoman algunos debates actuales sobre la estratificación social y la literatura sobre el tema, y sus vertientes en América Latina y Argentina, pero el manual enfatiza en el trabajo empírico con datos secundarios, la construcción de información, su análisis estadístico y sus posibilidades explicativas para interpretar cambios sociales recientes en Argentina, tanto desde un punto de vista global y regional, como también local.

El \protect\hyperlink{analisis1}{Capítulo 1} retoma algunos de los debates y conceptos en torno al estudio de la desigualdad y de la estructura social. Ya que una gran cantidad de publicaciones se dedican a este tema, se decidió dejarlo como introducción general conceptual, para pasar a los capítulos enfocados en los datos. El \protect\hyperlink{fuentes}{Capítulo 2} presenta las características generales de la \textbf{Encuesta Permanente de Hogares} (EPH), en tanto herramienta central para el estudio de la estructura de clases en Argentina, mientras que el \protect\hyperlink{fuentes2}{Capítulo 3} ofrece algo similar pero con los censos de población. El \protect\hyperlink{ocupacion}{Capítulo 4} se focaliza en ``las variables económicas'' de las fuentes descritas en los capítulos previos, y ya en \protect\hyperlink{clases5}{Capítulo 5} se presentan los diversos abordajes teórico-empíricos más utilizados para el estudio de la estratificación social, a nivel internacional y nacional, desde un abordaje operacional. El \protect\hyperlink{independiente}{Capítulo 6} presenta a las clases sociales como uno de los factores estructuradores de la desigualdad social. El \protect\hyperlink{dependiente}{Capítulo 7}, en cambio, retoma una de las posibles aproximaciones al estudio de la estructura de clases en tanto variable dependiente. Dentro de este tipo de abordajes se interroga acerca de aquellos factores que explican, condicionan o intervienen en el proceso de formación y acción de las clases sociales. Finalmente, en el \protect\hyperlink{tiempo}{Capítulo 8} se explora la dimensión temporal y la dimensión espacial, en tanto procesos sociales más importantes para evaluar los cambios que se producen en la estructura de clases y en la desigualdad. En el \protect\hyperlink{anexo}{Anexo}, se incluye una breve introducción a \textbf{\texttt{R}} y su interfaz \textbf{\texttt{RStudio}}, centrada en las funciones puntuales que permiten realizar los ejercicios de los capítulos.

Al socializar este trabajo, el anhelo es construir una comunidad de usuarios y que la información derivada del material presentado pueda llegar a contribuir al desarrollo de futuras investigaciones en profundidad por parte de los interesados. Aprovechando las ventajas de la publicación en línea, este escrito mantiene un formato ``vivo'', que se irá modificando con el tiempo y, si se da la posibilidad de la interacción con sus lectores, sus aportes serán felizmente bienvenidos. Porque pretender un manual omnicomprensivo y acabado de la \textbf{estructura social argentina} es un objetivo bastante utópico dada la enormidad de la literatura sobre el tema.

Esperamos que les sea de utilidad.

Nicolás, José y Sofía

\hypertarget{agradecimientos}{%
\section*{Agradecimientos}\label{agradecimientos}}
\addcontentsline{toc}{section}{Agradecimientos}

La idea de este manual fue tomando forma durante la pandemia de COVID-19 a mediados del año 2020, a partir de una invitación por parte de la \href{https://untref.edu.ar}{Universidad Nacional de Tres de Febrero (UNTreF)} para elaborar un curso de posgrado, en particular, de Jorge Fernández Bussy, Director de la Maestría en Generación y Análisis de Información Estadística, en el marco del Ciclo de Seminarios y Cursos extracurriculares en Estadística y el Programa de Actualización Permanente de las Orientaciones de Estadísticas Económicas, Estadísticas Sociodemográficas y Estadísticas de Opinión y Mercado, en coordinación con la Secretaría de Extensión Universitaria y la Dirección de Posgrado. El curso ``Taller de Análisis de la Estructura Social en la Argentina Contemporánea'' fue ofrecido por les autores de este libro de formato a distancia híbrdo (a-sincrónico y sincrónico) durante el segundo semestre del 2020. Queremos por ello agradecer a Jorge y al personal de la UNTreF por todo su apoyo, quienes permitieron llevar a cabo con éxito esta experiencia. También de forma especial a les estudiantes que participaron durante esos tiempos tan difíciles, que con su trabajo y aportes fueron nuestra primera audiencia. Muchas gracias tambín a Gabriela Benza, Ezequiel Adamovsky y Eduardo Chávez Molina, quienes participaron como entrevistados especialistas del tema en un módulo del curso.

\hypertarget{part-aspectos-teuxf3ricos}{%
\part{Aspectos teóricos}\label{part-aspectos-teuxf3ricos}}

\hypertarget{estructura1}{%
\section{La estructura social en la Argentina actual}\label{estructura1}}

Este capítulo propone acercamiento introductorio al tema de las clases sociales, a través del abordaje de algunos conceptos básicos del estudio de la desigualdad social y de la estructura social, las tendencias actuales y cómo ésta subdisciplina ha evolucionado a través del tiempo.

Optamos por organizar un capítulo introductorio a modo de repaso de algunos conceptos básicos, resaltando que todo contenido empírico tiene un trasfondo teórico que es necesario conocer y para ello recomendamos profundizarlos a través de cursos, talleres o lecturas específicas. Si bien este manual no se recuesta sobre estos aspectos, consideramos imprescindible tenerlos en cuenta tanto al momento de construcción como en el análisis de los datos, ya que si no, se corre el riesgo de hacer ``investigación social en un termo'', como dicen por ahí.

El objetivo de este primer capítulo es exponer los principales debates teóricos de ``estructura social'' a manera de ir delineando definiciones teóricas y operacionales de clases sociales. Las preguntas guías remiten a considerar: ¿Cómo distinguir la clase? ¿Cómo encajan distintas definiciones en el debate actual de las clases sociales? La idea es también ver cómo en las formulaciones clásicas sobre la temática se puede volver a abrir el debate sobre ``la muerte de las clases'' y cuáles son las implicaciones de esta problemática para la investigación empírica de la sociedad argentina. Se muestra asimismo cómo el cuestionamiento de la estratificación social es un problema no tan nuevo pero inagotable y se esforzará por describir porqué la investigación social sobre clases está lejos de haber perdido su relevancia.

El capítulo se divide en cinco partes. En la primera se trazan a grandes rasgos los principales debates sobre las clases sociales y su eco en América Latina. Luego distinguimos algunos de los marcos teóricos posibles de aplicar para la construcción de preguntas acerca de la desigualdad social. En tercer lugar, buscamos explicar la importancia del estudio de la composición y estructura de la población según clases sociales y sus implicaciones para las políticas públicas, para luego pasar a identificar las tendencias históricas de su estudio en Argentina.

\hypertarget{analisis1}{%
\subsection{Del análisis de las clases sociales a la teoría de la estratificación social}\label{analisis1}}

¿Cuáles son las dimensiones de la estructura social? ¿Qué son las clases sociales? Los criterios empleados para la definición metodológica y operacional de las clases sociales y las interpretaciones actuales sobre estructura social en la Argentina reconocen como referencia aportes de diversos enfoques teóricos. Aquí se repasan aquellos que han sido revisados ampliamente en diferentes publicaciones, ya que la acumulación de literatura sobre el tema es extensa. Si bien se hace un recorrido histórico por la cuestión, estos debates pueden esquemáticamente reunirse en:

\begin{itemize}
\tightlist
\item
  los postulados teóricos del materialismo histórico y el neomarxismo;
\item
  los planteamientos sociológicos de Weber y de los neoweberianos;
\item
  el enfoque estructural-funcionalista;
\item
  el enfoque empirista-estadístico o pragmático y
\item
  los enfoques de ingresos.
\end{itemize}

Esta reconstrucción, comprensiblemente, no es sencilla ya que la herencia teórica de ciento cincuenta años de sociología deja muchas propuestas y variaciones en torno a un tema común y una bibliografía extraordinaria. El resumen sin duda no excluye la existencia de otros diseños y sería en vano enunciar todos como así buscar la formulación definitiva sobre esta cuestión.

Existe una amplia literatura dedicada al análisis de la estructura social, de la cual destacamos dos puntos en común, que nos ayudan a establecer un punto de partida para su definición (\protect\hyperlink{ref-FeitoAlonso1995}{Feito Alonso, 1995}). El primer punto alude al \emph{carácter temporal} de la estructura social, ya que hace referencia a las relaciones estables de la sociedad, es decir, que describe las regularidades de sus elementos constituyentes. El segundo punto refiere a su \emph{aspecto plural}, que designa los rasgos de grupos y sociedades, en otras palabras, que excede a los aspectos individuales.

De alguna manera existe una especie de conocimiento ``natural'' sobre la estratificación social que en general constituye una cuestión en la que reside una buena parte de la dificultad de su análisis. No es el objetivo de este manual hacer una historia de la lectura sociológica sobre las clases sociales, y menos sobre la estratificación social. Esa tarea es demasiado ambiciosa para un objetivo que está puesto en otro lado. Las preguntas-guía señaladas anteriormente remiten a delimitar los términos que habitualmente que se utilizan en el análisis de las clases sociales y la estratificación para entender su relación con el resto de los capítulos de este libro.

En este sentido, tres son las dimensiones relevantes de la estructura social (\protect\hyperlink{ref-Carabana1997}{Carabaña, 1997}): la económica, la social y la cívica-política. En este libro, al referirnos a las clases sociales, apuntaremos al abordaje de la dimensión socio-económica de la estructura social. De esta forma, estructura social y estructura de clases, si bien muchas veces son utilizados como sinónimos, no son términos intercambiables entre sí. De forma resumida, partiendo del supuesto de que la sociedad es desigual, se entiende a la estructura social como el concepto que designa a las estructuras que conforman la sociedad. La estructura de clases es una forma específica de estratificación social. A continuación proponemos una guía resumen de los principales enfoques que han contribuido al análisis de las clases sociales.

\hypertarget{marx}{%
\subsection{Marx, Weber y el funcionalismo}\label{marx}}

Tres grandes tradiciones se inscriben en el debate sociológico actual sobre la definición, medición y operacionalización del concepto de ``clase'': son los desarrollos teóricos de los ``clásicos'' de la disciplina, Marx y Weber, y la teoría funcionalista de la estratificación social (\protect\hyperlink{ref-Crompton2008}{Crompton, 2008}). La concepción marxista sobre las clases sociales fue influenciada por importantes perspectivas teóricas: la filosofía alemana, la economía política inglesa y el socialismo utópico alemán (\protect\hyperlink{ref-Ossowski1972}{Ossowski, 1972}). Crompton (\protect\hyperlink{ref-Crompton2008}{2008, p. 11}) ubicando al concepto de ``clase'' como herramienta de análisis social ligado al desarrollo del capitalismo y a la transición a la modernidad, señaló que Marx fue el primer teórico en considerar la naturaleza económica (o ``materialista'') de la desigualdad social.

En efecto, Marx fue quien vinculó el concepto de clase social con el de modo de producción y a las clases sociales con etapas históricas de los modos de la producción. Según esta corriente de pensamiento las clases sociales se constituyen a partir de la estructura económica del modo de producción capitalista, y el lugar que ocupan las personas en el proceso de producción y su relación con los medios de producción el determinante de su posición social.

Al situar a la lucha de clases como motor de la historia, Marx concedía al concepto de clase una distinción científica y, paralelamente, le asignaba un rol explicativo de la sociedad y la historia. El término de \textbf{clase} en su obra puede encontrarse tanto de forma analítica como descriptiva, es decir, como un concepto históricamente determinado. Wright (\protect\hyperlink{ref-Wright1997}{1997}) señaló que Marx enfrentó el problema abstracto de las clases y, al mismo tiempo, se dedicó al análisis concreto de las clases como actores, es decir, en distintos niveles de análisis. Las nociones de ``clase en sí'' y ``clase para sí'' ---por un lado, entre la existencia de la clase como una realidad histórica y, por el otro, la clase como consciente de su identidad y de capacidad de actuar--- motivaron un extenso debate sobre el cual, a lo largo del siglo XX, el materialismo histórico se desenvolvió en diversas corrientes: la ``humanista'', que enfatizó la significancia de la acción en las explicaciones sobre las instituciones y el comportamiento (Gramsci) y la ``científica'' o ``estructuralista'' (Althusser, Poulantzas) que resaltó la descripción de ``estructuras'' de clase (\protect\hyperlink{ref-Crompton2008}{Crompton, 2008, p. 31}).

La premisa que considera a las clases como la variable explicativa del fenómeno de las desigualdades sociales tuvo en Marx el eje sobre el cual giró su discusión teórica, incluso hasta el presente. Su obra fue puesta en ``diálogo'' por Weber, quien criticó el principio de propiedad determinante de la clase social. Weber (\protect\hyperlink{ref-Weber2002}{2002, pp. 682--694}) propuso la existencia de una multiplicidad de factores que van más allá de la propiedad: estatus, relaciones de mercado, habilidades, prestigio, extendiendo de esta manera la dimensión ``única'' de clase. Considerando las ideas de Weber una extensión de las de Marx, a pesar de sus diferencias, el contraste entre estos dos autores puede, a veces, parecer exagerado. Weber no dejó de problematizar la visión materialista de la historia en Marx, ampliando esa visión ``económica'' y elaborando una propuesta multidimensional para el análisis del fenómeno de la estratificación social, distinguiendo a las clases, los estamentos y los partidos, donde la acción social predomina sobre una estructura objetiva, como el modo de producción (\protect\hyperlink{ref-Burris1995}{Burris, 1995, p. 130}).

Para Kerbo (\protect\hyperlink{ref-Kerbo2003}{2003}) ---quien distinguió de las teorías de la estratificación social las ``teorías del conflicto'' y ``teorías funcionales de la sociedad'' (resumidas en la Tabla \ref{tab:textkerbo} ---en lugar ``de los supuestos críticos del conflicto en Marx, en Weber encontramos un conjunto de supuestos paradigmáticos que hemos denominado no críticos del conflicto'' (p.~95). Si bien ``Weber creía que el conflicto estaba más extendido y se encontraba en el núcleo mismo de la organización social compleja, no albergaba, a diferencia de Marx, la esperanza de que este conflicto humano se pudiera eliminar por completo algún día'' (p.~107).

\providecommand{\docline}[3]{\noalign{\global\setlength{\arrayrulewidth}{#1}}\arrayrulecolor[HTML]{#2}\cline{#3}}

\setlength{\tabcolsep}{2pt}

\renewcommand*{\arraystretch}{1.5}

\begin{longtable}[c]{cccc}

\caption{Tipología de paradigmas de la estratificación social
}\label{tab:textkerbo}\\

\hhline{>{\arrayrulecolor[HTML]{666666}\global\arrayrulewidth=2pt}->{\arrayrulecolor[HTML]{666666}\global\arrayrulewidth=2pt}->{\arrayrulecolor[HTML]{666666}\global\arrayrulewidth=2pt}->{\arrayrulecolor[HTML]{666666}\global\arrayrulewidth=2pt}-}

\multicolumn{1}{!{\color[HTML]{000000}\vrule width 0pt}>{}l}{\fontsize{10}{10}\selectfont{\textcolor[HTML]{000000}{\textbf{}}}} & \multicolumn{1}{!{\color[HTML]{000000}\vrule width 0pt}>{}l}{\fontsize{10}{10}\selectfont{\textcolor[HTML]{000000}{\textbf{}}}} & \multicolumn{1}{!{\color[HTML]{000000}\vrule width 0pt}>{}l}{\fontsize{10}{10}\selectfont{\textcolor[HTML]{000000}{\textbf{Críticos}}}} & \multicolumn{1}{!{\color[HTML]{000000}\vrule width 0pt}>{}l!{\color[HTML]{000000}\vrule width 0pt}}{\fontsize{10}{10}\selectfont{\textcolor[HTML]{000000}{\textbf{No\ críticos}}}} \\

\hhline{>{\arrayrulecolor[HTML]{666666}\global\arrayrulewidth=2pt}->{\arrayrulecolor[HTML]{666666}\global\arrayrulewidth=2pt}->{\arrayrulecolor[HTML]{666666}\global\arrayrulewidth=2pt}->{\arrayrulecolor[HTML]{666666}\global\arrayrulewidth=2pt}-}

\endfirsthead

\hhline{>{\arrayrulecolor[HTML]{666666}\global\arrayrulewidth=2pt}->{\arrayrulecolor[HTML]{666666}\global\arrayrulewidth=2pt}->{\arrayrulecolor[HTML]{666666}\global\arrayrulewidth=2pt}->{\arrayrulecolor[HTML]{666666}\global\arrayrulewidth=2pt}-}

\multicolumn{1}{!{\color[HTML]{000000}\vrule width 0pt}>{}l}{\fontsize{10}{10}\selectfont{\textcolor[HTML]{000000}{\textbf{}}}} & \multicolumn{1}{!{\color[HTML]{000000}\vrule width 0pt}>{}l}{\fontsize{10}{10}\selectfont{\textcolor[HTML]{000000}{\textbf{}}}} & \multicolumn{1}{!{\color[HTML]{000000}\vrule width 0pt}>{}l}{\fontsize{10}{10}\selectfont{\textcolor[HTML]{000000}{\textbf{Críticos}}}} & \multicolumn{1}{!{\color[HTML]{000000}\vrule width 0pt}>{}l!{\color[HTML]{000000}\vrule width 0pt}}{\fontsize{10}{10}\selectfont{\textcolor[HTML]{000000}{\textbf{No\ críticos}}}} \\

\hhline{>{\arrayrulecolor[HTML]{666666}\global\arrayrulewidth=2pt}->{\arrayrulecolor[HTML]{666666}\global\arrayrulewidth=2pt}->{\arrayrulecolor[HTML]{666666}\global\arrayrulewidth=2pt}->{\arrayrulecolor[HTML]{666666}\global\arrayrulewidth=2pt}-}\endhead



\multicolumn{4}{!{\color[HTML]{FFFFFF}\vrule width 0pt}>{}l!{\color[HTML]{FFFFFF}\vrule width 0pt}}{\fontsize{10}{10}\selectfont{\textcolor[HTML]{000000}{Fuente:\ Kerbo\ (2003,\ p.\ 83)}}} \\

\endfoot



\multicolumn{1}{!{\color[HTML]{000000}\vrule width 0pt}>{}l}{} & \multicolumn{1}{!{\color[HTML]{000000}\vrule width 0pt}>{}l}{\fontsize{10}{10}\selectfont{\textcolor[HTML]{000000}{Orden}}} & \multicolumn{1}{!{\color[HTML]{000000}\vrule width 0pt}>{}l}{\fontsize{10}{10}\selectfont{\textcolor[HTML]{000000}{Paradigma\ crítico\ del\ orden}}} & \multicolumn{1}{!{\color[HTML]{000000}\vrule width 0pt}>{}l!{\color[HTML]{000000}\vrule width 0pt}}{\fontsize{10}{10}\selectfont{\textcolor[HTML]{000000}{Paradigma\ no\ crítico\ -\ Teoría\ funcional\ (Durkheim)}}} \\





\multicolumn{1}{!{\color[HTML]{000000}\vrule width 0pt}>{}l}{\multirow[c]{-2}{*}{\fontsize{10}{10}\selectfont{\textcolor[HTML]{000000}{Modelo\ de\ sociedad}}}} & \multicolumn{1}{!{\color[HTML]{000000}\vrule width 0pt}>{}l}{\fontsize{10}{10}\selectfont{\textcolor[HTML]{000000}{Conflicto}}} & \multicolumn{1}{!{\color[HTML]{000000}\vrule width 0pt}>{}l}{\fontsize{10}{10}\selectfont{\textcolor[HTML]{000000}{Paradigma\ crítico\ del\ conflicto\ -\ Teoría\ de\ la\ clase\ dominante\ (Marx)}}} & \multicolumn{1}{!{\color[HTML]{000000}\vrule width 0pt}>{}l!{\color[HTML]{000000}\vrule width 0pt}}{\fontsize{10}{10}\selectfont{\textcolor[HTML]{000000}{Paradigma\ no\ crítico\ -\ Teoría\ del\ conflicto\ (Weber)}}} \\

\hhline{>{\arrayrulecolor[HTML]{666666}\global\arrayrulewidth=2pt}->{\arrayrulecolor[HTML]{666666}\global\arrayrulewidth=2pt}->{\arrayrulecolor[HTML]{666666}\global\arrayrulewidth=2pt}->{\arrayrulecolor[HTML]{666666}\global\arrayrulewidth=2pt}-}



\end{longtable}

Dentro del paradigma no crítico y pese al poco desarrollo teórico acerca de la problemática de las clases sociales, como antecedente del funcionalismo en sociología, las ideas de Durkheim se ubicaron en contraste con la teoría del conflicto de Marx o Weber. Sus propuestas fueron de enorme ayuda para los teóricos funcionalistas y su interpretación de las clases sociales, ya que \textgreater{}``calificó la existencia de la clase y el conflicto de clase de no naturales{[}\ldots{]} esta condición patológica del conflicto existía porque los grupos ocupacionales no cumplían adecuadamente la función de proporcionar orden moral y porque los intereses egoístas de individuos y grupos amenazaban a la sociedad. Pero a Durkheim nunca se le pasó por la mente que todo un sistema de división del trabajo en la sociedad industrial pudiera ser una estructura de poder al servicio de la dominación de una clase por otra (como mantienen los teóricos del conflicto)'' {[}\ldots{]} (\protect\hyperlink{ref-Kerbo2003}{Kerbo, 2003, p. 106}).

La obra estructural-funcionalista de Parsons, que predominó en la teoría sociológica después de la II Guerra Mundial, reflejó ciertos elementos del análisis de Durkheim sobre las consecuencias de la división del trabajo en la ``sociedad industrial''. En base a ella, las teorías funcionales -como por ejemplo las de Davis \& Moore (\protect\hyperlink{ref-Levine2006}{Levine, 2006})- de la estratificación propusieron que las desigualdades en las sociedades complejas se imponen como legítimas a través de un consenso de valores relacionados con la importancia social de funciones particulares (\protect\hyperlink{ref-Crompton2008}{Crompton, 2008, p. 13}).

\hypertarget{esperanza}{%
\subsection{La esperanza de vida de la teoría: debates contemporáneos}\label{esperanza}}

Después del \emph{big bang} teórico de clásicos, en la década del cuarenta y el cincuenta del siglo XX predominó lo que algunos autores llamaron el ``consenso ortodoxo'' del análisis estructural-funcionalismo de la estratificación social, de fuerte tradición estadounidense y en las antípodas de la ``teorías del conflicto'' (\protect\hyperlink{ref-Martinez2005}{Martínez, 2005, p. 47}). Relacionados, en parte, con los conflictos sociales y políticos de la década del sesenta, surgieron diversos enfoques analíticos en la sociología norteamericana cuyo cuestionamiento alcanzó a la academia. En particular, estas propuestas ponían en duda la capacidad del funcionalismo para comprender esos fenómenos sociales y ello contribuyó a fundamentar los planteos críticos no solo a sus debilidades teóricas, sino también a sus implicaciones políticas e ideológicas. Ante las propuestas de Parsons, Merton y discípulos, distintos autores contribuyeron a minar ese ``consenso ortodoxo''.

Gouldner (\protect\hyperlink{ref-Gouldner2000}{2000}) caracterizó esa etapa como la de una situación de crisis de la sociología. Ante la propagación de enfoques alternativos y material empírico que socavaba las bases teóricas del funcionalismo, la fragmentación de perspectivas no dio lugar ---en el mediano plazo--- a un marco teórico amplio capaz de cubrir los vacíos dejados por ``la gran teoría''. Estos debates tuvieron influencia en el desarrollo teórico de la estratificación social.

Si bien luego de la segunda posguerra no abundó en el materialismo histórico una reflexión sustantiva que aportara nuevos aspectos al concepto de clase (\protect\hyperlink{ref-FeitoAlonso1995}{Feito Alonso, 1995, p. 55}), a comienzos de los sesenta el panorama cambió. Hubo una respuesta ``a la europea'' al funcionalismo de parte de autores como Lockwood (\protect\hyperlink{ref-Lockwood1962}{1962}) y Dahrendorf (\protect\hyperlink{ref-Dahrendorf1979}{1979}), quienes se esforzaron por relacionar el legado de Marx y Weber (\protect\hyperlink{ref-FeitoAlonso1995}{Feito Alonso, 1995, pp. 50--51}) en los estudios de estratificación social, mientras que desde Francia comenzaban las relecturas del marxismo de Althusser (\protect\hyperlink{ref-Althusser1968}{1968}). Para Parkin

\begin{quote}
``\ldots inesperadamente el marxismo contemporáneo se ha aproximado a la posición sociológica. Este deslizamiento sobrevino como parte de un intento más general de los marxistas occidentales en procura de reconsiderar el modelo clásico u ortodoxo de las clases bajo las nuevas condiciones del capitalismo monopolista\ldots{}'' (\protect\hyperlink{ref-Parkin1968}{1968, pp. 696--697}).
\end{quote}

Fue también a mediados de los setenta del siglo XX cuando el revisionismo ``neomarxista'' de Poulantzas (\protect\hyperlink{ref-Poulantzas2005}{2005}) abrió un intenso debate en las ciencias sociales. Su esfuerzo teórico intentó definir los conceptos claves para el análisis de las clases sociales desde la perspectiva del materialismo histórico. Poulantzas sostuvo que dentro del proceso de producción, la dimensión económica distingue al trabajo productivo del improductivo, la dimensión política refiere al grado de control sobre el proceso de trabajo, mientras que la dimensión ideológica manifiesta el grado de conciencia y participación, o no, de la organización del proceso de trabajo (\protect\hyperlink{ref-Parkin1968}{Parkin, 1968}).

\hypertarget{la-grieta-de-los-ochenta}{%
\subsubsection{La grieta de los ochenta}\label{la-grieta-de-los-ochenta}}

Para principios de los '80, el debate sobre las clases sociales continuó dentro del materialismo histórico, si bien su perspectiva ``estructuralista'' no tuvo mayor influencia; el colapso del bloque socialista y la llegada del neoconservadurismo al poder en los países centrales minó de cierta manera el optimismo de izquierda de los sesenta y setenta y la discusión giró en torno a las posibilidad del socialismo en ese contexto ---por ejemplo, Laclau (\protect\hyperlink{ref-Laclau1987}{1987})---. Este diálogo, si bien no estaba directamente relacionado con la investigación en estratificación social en la sociología académica, impactó en los debates de ese momento.

Es que, paradójicamente, es en la década de 1980 cuando el enfoque del análisis de clase asistió a un momento de gran creación, tanto en cuestiones conceptuales como de análisis empírico. Producto del ``retorno'' a los clásicos, el desarrollo de las propuestas de Wright y Goldthorpe se produjo en paralelo con una sofisticación de las técnicas de investigación (sobre todo en lo que refiere al procesamiento de datos y al análisis estadístico). Los enfoques teóricos y las propuestas de aplicación de investigación de Wright y Goldthorpe fueron descritos por Crompton (\protect\hyperlink{ref-Crompton2008}{2008, p. 50}) como el enfoque ``ocupacional-agregado'' al que considera una orientación especializada del análisis de clase en el campo de la sociología. Para Crompton las distintas clasificaciones y estrategias para el estudio empírico de la estructura de clase u ocupacional pueden dividirse en tres:

\begin{enumerate}
\def\labelenumi{\roman{enumi}.}
\tightlist
\item
  esquemas de ``sentido común'' con pocas pretensiones teóricas, que agrupan ocupaciones en un orden jerárquico con líneas de corte ``objetivas'';
\item
  índices de prestigio ocupacional o \emph{status}, que buscan medir el ranking social de distintas ocupaciones (frecuente, sobre todo, en las investigaciones sobre movilidad social);
\item
  esquemas ``relacionales'' o ``formas teóricas de esquemas de clases'', que poseen explícitas referencias a las propuestas teóricas de Marx y Weber.
\end{enumerate}

Los esquemas i) y ii), Crompton los denominó ``índices descriptivos'' que incluyen los ``gradacionales'' o ``jerárquicos'' donde las clases difieren por el grado cuantitativo de algún atributo o basados en una cualidad particular (ingresos, prestigio, status, etc.) . Los agrupados iii) persiguen, en líneas generales, la descripción de la desigualdad ocupacional, determinando teóricamente las ``clases'' y a nivel empírico, observando las relaciones de clase ---refiriendo con ello el proceso por el cual esas desigualdades se producen--- (\protect\hyperlink{ref-Crompton2008}{2008, p. 49}). Se incluyen aquí los dos grandes proyectos que dominaron el debate en el campo del análisis de clase durante los ochenta y los noventa: el \emph{Compartive Class Project} ``neo-marxista'' de Wright y el \emph{CASMIN}\footnote{Comparative Analysis of Social Mobility in Industrializad Countries.} ``neo-weberiano'' de Goldthorpe.

Así como es frecuente comentar que Weber ``dialoga'' con la obra de Marx, Burris (\protect\hyperlink{ref-Burris1995}{1995}) sugirió que la literatura marxista contemporánea ``dialoga'' con Weber. Por ejemplo, Wright insistió en formalizar un desarrollo teórico manteniendo a la vez un esfuerzo de aplicación, empírico, bajo el objetivo general de ``construir, dentro de un marco teórico marxista en sentido amplio, un concepto de estructura de clases susceptible de ser usado en el análisis de procesos micro a un nivel relativamente bajo de abstracción'' (1995:21). En base a ello elaboró dos estrategias metodológicas:

\begin{quote}
``he explorado dos enfoques generales diferentes de este problema {[}estructura de clases{]}, a los que podríamos denominar respectivamente enfoque de las posiciones contradictorias y enfoque de la exploración multidimensional. Ambas estrategias constituyen intentos de proporcionar una teorización positiva a la categoría''clase media'' dentro de un marco esencialmente basado en los intereses.'' (\protect\hyperlink{ref-Wright1995a}{Wright, 1995, p. 59}).
\end{quote}

La primera solución de Wright para ubicar a la clase media fue considerarla entre la clase obrera y la pequeña burguesía, simultáneamente, tomando este concepto de posición contradictoria. La segunda solución consideraba que

\begin{quote}
``los diferentes `modos de producción' se basan en mecanismos específicos de explotación que pueden diferenciarse según el tipo de recurso productivo, cuya desigual propiedad (o control) permite a la clase explotadora apropiarse parte del excedente socialmente producido. Basándome en la obra de Roemer, distinguía yo cuatro tipos de recursos, la desigual propiedad o control de los cuales constituía la base de las distintas formas de explotación: los bienes de fuerza de trabajo (explotación feudal), los bienes de capital (explotación capitalista), los bienes de organización (explotación estatista) y los bienes de cualificación o credenciales (explotación socialista). Aunque los modos puros de producción pueden identificarse con las formas simples de explotación, las sociedades reales siempre constan de diferentes formas de combinación de los diferentes mecanismos de explotación. Esto abre la posibilidad de que ciertas posiciones en la estructura de clases estén simultáneamente explotadas a través de un mecanismo de explotación pero sean explotadoras a través de otro mecanismo''. Dichas posiciones\ldots constituyen la `clase media' de una sociedad dada'' (\protect\hyperlink{ref-Wright1995a}{Wright, 1995, pp. 65--66}).
\end{quote}

\hypertarget{nuevos-debateshacia-una-operacionalizaciuxf3n-muxe1s-refinada-de-los-conceptos}{%
\subsubsection{Nuevos debates\ldots¿Hacia una operacionalización más refinada de los conceptos?}\label{nuevos-debateshacia-una-operacionalizaciuxf3n-muxe1s-refinada-de-los-conceptos}}

Este modelo, durante buena parte de la década del noventa, estuvo un tiempo ``encontrado'' a las líneas de pensamiento ``neoweberianas'' de los cuales resalta el trabajo de Goldthorpe y que incluyen los de Giddens y Parkin. Con una influencia importante en los estudios sobre estratificación social en América Latina, Goldthorpe retomó la definición de clase desde la estructura ocupacional, diferenciando la ``situación de trabajo'' y la ``situación de mercado'', utilizando la ocupación a partir de una escala de prestigio ocupacional. El autor trató de encontrar la ``identidad demográfica'' de clase y el ``grado de formación sociopolítica de la clase'' (\protect\hyperlink{ref-Martinez2005}{Martínez, 2005, p. 74}). Trabajó en el análisis de la ``clase de servicio'' ---profesionales, por un lado, administrativos y directivos por el otro--- que, una vez despejados sus rasgos estructurales (económicos), intentó delinear su paso de las ``clases económicas'' a las clases sociales.

Crompton también ubicó dentro del enfoque ``ocupacional-agregado'' al \emph{Clasificador Estadístico Socio-Económico Nacional}(NS-SEC) de la Oficina de Estadística del Reino Unido, derivado, en mayor parte, del esquema clasificatorio de Goldthorpe, que combina el \emph{status} en el empleo con características del trabajo. Se trata de un sistema clasificatorio que utiliza el sistema estadístico inglés y que se incluye dentro del programa de desarrollo de un ``sistema de clasificación socio-económico europeo'' (\protect\hyperlink{ref-Rose2007}{Rose \& Harrison, 2007}). Es que, en efecto, el más frecuente uso ``dado por hecho'' en el análisis contemporáneo de clases sociales en sociología es aquel que divide grupos ordinales de ocupaciones en ``clases'' (\protect\hyperlink{ref-Crompton2008}{Crompton, 2008, p. 49}) y la división de la población en posiciones desiguales de recompensas es comúnmente descripta como ``estructura de clase''. Noción que, en la actualidad, se focaliza usualmente en la estructura del empleo.

La cuestión acerca de cómo se utiliza a la variable ocupación en términos de clasificación tiene distintas respuestas (\protect\hyperlink{ref-Desrosieres2000}{Desrosières \& Thévenot, 2000}). Ganzeboom (\protect\hyperlink{ref-Ganzeboom1992}{1992}), por ejemplo, distingue dos corrientes entre los científicos sociales: los que utilizan aproximaciones a la clasificación de categorías de clase socioeconómica y los que prefieren medidas continuas. El uso de la variable ocupación es columna vertebral de este tipo de orientación. Para los sociólogos ---y también para las agencias comerciales y organismos del gobierno--- la ocupación se incorporó como variable principal para el análisis de clase ya sea porque era el indicador más poderoso de recompensas materiales o por la disponibilidad de información con la que se contaba para realizar investigaciones con fundamento empírico.

Como señala Barozet (\protect\hyperlink{ref-Barozet2007}{2007}) los datos de ocupación fueron los predilectos para los estudios de estratificación social según este enfoque, abordado desde distintos puntos de vista teóricos y metodológicos. Esta autora, haciendo un recorrido por las escalas académicas internacionales que utilizaron esta variable ---la \emph{ISCO}(International Standard Classification of Occupations); la de Goldthorpe y Erikson; la ya mencionada de Eric Olin Wright; la escala de ocupación francesa y por aquellas que trabajaron en base al prestigio ocupacional, como las de la de Treiman (\emph{Standard International Occupational Prestige Scale}) (\protect\hyperlink{ref-Treiman1977}{Treiman, 1977})-- mencionó que:

\begin{quote}
``La variable ocupación es central para los estudios de estratificación social, pero cabe tener mucho cuidado de que el uso que se le da no aparte los estudios de una posible comparación. En efecto, siguiendo la advertencia de Ganzeboom y Treiman (1996), es importante tomar como piso la clasificación ISCO, con el fin de establecer ocupaciones que se puedan subsumir a categorizaciones internacionales. Sin embargo, el uso de la variable ocupación se enfrenta al mismo problema que otras variables: las ocupaciones, sus nombres y sus contenidos cambian, por lo cual se requiere de una adaptación en cada nueva aplicación de encuesta'' {[}Barozet (\protect\hyperlink{ref-Barozet2007}{2007}), p.~45).
\end{quote}

\hypertarget{no-tan-distintos}{%
\paragraph{No tan distintos}\label{no-tan-distintos}}

Con varios puntos en común y, llamativamente, a pesar del disenso teórico, con resultados empíricos homologables al menor nivel de apertura de sus esquemas (\protect\hyperlink{ref-Crompton2008}{Crompton, 2008}), las propuestas teóricas Goldthorpe y Wright relacionan, por un lado, la clase como principio estructural de desigualdad social y, por el otro, la clase como origen de identidad social y política, conciencia y acción, cuestión en la que reside una gran fuente de debates.

Desde la ``etnografía de clase'' se defendió la idea que la estructura y la acción, la economía y la cultura, tienen roles importantes ---e intervienen--- en el mantenimiento y la reproducción de clases sociales. Este punto de vista argumentó que no sólo las diferencias económicas sino también mecanismos no-económicos se vinculan con las desigualdades sociales y la cultura, es decir, que al observar la estructura social no se trata solamente de separar el efecto de la localización en clases sociales sino también hay que tener en cuenta los mecanismos por los cuales las mismas clases se constituyen. En este segundo tipo de interpretación se puede ubicar la teoría de la estructuración de Giddens y el trabajo de Bourdieu. Es que las críticas al análisis de clase dentro de la sociología transitaron por diversos andariveles: se suele problematizar una supuesta poca capacidad para explicar la acción de clase y su ``sobredeterminación economicista''. Al mismo tiempo, dentro de las posibles definiciones de clase que pueden ser identificadas en la literatura especializada, se observó un creciente interés general por su dimensión cultural (\protect\hyperlink{ref-Crompton2008}{Crompton, 2008, p. 16}) con un ``giro'' de perspectiva sobre el abordaje más tradicional de estructura de clase en base a la ocupación.

Por un lado, Giddens (\protect\hyperlink{ref-Giddens1998}{1998}) propuso relacionar las trayectorias conceptuales provenientes de Europa y Estados Unidos. En su ``teoría de la estructuración'' debatió la ``concepción utópica'' y el ``evolucionismo de la historia'' en el marxismo, la poca importancia concedida al Estado en el ejercicio del poder y la desatención a formas no económicas de desigualdad social. En su crítica al ``objetivismo determinista'' del funcionalismo y del marxismo reclamó un enfoque que dejara las intenciones de los actores en el mismo nivel que el de las estructuras. Por otro lado Bourdieu, en cambio, enfatizó en los aspectos económicos y culturales de la diferenciación social, influido tanto por una visión marxista como por la weberiana. En \emph{La distinción} (\protect\hyperlink{ref-Bourdieu2012}{2012}) no redujo el análisis de clase a su dimensión económica sino que, al mismo tiempo, analizó relaciones sociales simbólicas. Para este autor la clase social

\begin{quote}
``no se define por una propiedad {[}\ldots{]} ni por una suma de propiedades {[}\ldots{]} ni mucho menos por una cadena de propiedades ordenadas a partir de una propiedad fundamental (la posición en las relaciones de producción) en una relación de causa a efecto, de condicionante a condicionado, sino por la estructura de las relaciones entre todas las propiedades pertinentes, que confiere su propio valor a cada una de ellas y a los efectos que ejerce sobre las prácticas\ldots{}'' (\protect\hyperlink{ref-Bourdieu2012}{Bourdieu, 2012, p. 121}).
\end{quote}

Distintas combinaciones de capital (económico, social, cultural o simbólico) constituyen el \emph{habitus}, que \textgreater{}``es a la vez, en efecto, el \emph{principio generador} de prácticas objetivamente enclasables y el \emph{sistema de enclasamiento (principium divisionis)} de esas prácticas.'' (\protect\hyperlink{ref-Bourdieu2012}{Bourdieu, 2012, p. 200}).

En paralelo, desde el paradigma weberiano contemporáneo algunos autores se empeñaron en cuestionar la dimensión estructuralista del materialismo histórico, que deja de lado el rol de la acción social y la primacía de clase por sobre otras formas no clasistas de dominación. Según señaló Burris (\protect\hyperlink{ref-Burris1995}{1995, p. 131}), Giddens y Parkin ignoraron una parte del debate marxista contemporáneo cuando argumentaron que los fundamentos marxistas sostenidos en la lógica de los modos de producción son una forma de funcionalismo y que el marxismo reduce a los actores a espectadores pasivos de sus relaciones sociales. Es que en el materialismo histórico ha ``corrido mucha agua bajo el puente'' desde las explicaciones clásicas de Marx para comprender los mecanismos de las instituciones o de las prácticas que operan sobre la estructura social.

En buena parte de la literatura marxista contemporánea se asignó un papel significativo a la acción humana en las explicaciones y en el análisis de clase. Incluso, como bien señala Burris, pueden encontrarse marxistas como Thompson (\protect\hyperlink{ref-Thompson1968}{1968}) o Przeworski (\protect\hyperlink{ref-Przeworski1977}{1977}), que, aunque sitúen la primacía de las estructuras por sobre la acción, no implica que la dejen de lado (\protect\hyperlink{ref-Burris1995}{Burris, 1995, p. 133}). De hecho, Thompson analizó las clases sociales en términos de acción humana casi exclusivamente, en su clásico estudio sobre la formación histórica de la clase obrera en Inglaterra. Su concepción de clase como proceso y relación ---si bien nunca detalló una teoría sistemática de clase--- se contrapone al de clase como ubicación estructural:

\begin{quote}
``Por clase entiendo un fenómeno histórico que unifica a un número de sucesos dispares y aparentemente desconectados, tanto en la materia prima de la experiencia como en la consciencia. No veo a la clase como una''estructura'', ni siquiera como una ``categoría'', sino como algo que de hecho acontece (y puede demostrarse que ha acontecido) en las relaciones humanas {[}\ldots{]}. La clase se define por cómo viven los hombres su propia historia y, en última instancia, ésta es su única definición'' (\protect\hyperlink{ref-Thompson1968}{Thompson, 1968, pp. 9--11}).
\end{quote}

Przeworski equiparó la importancia de la estructura objetiva y la acción como mutuamente dependientes cuando ``analiza las clases como el resultado de las luchas históricas concretas condicionadas por estructuras sociales que, a su vez, retroactúan sobre esas estructuras y las transforman'' (\protect\hyperlink{ref-Burris1995}{Burris, 1995, p. 133}).

\hypertarget{la-muerte-de-la-clases}{%
\subsubsection{La muerte de la clases}\label{la-muerte-de-la-clases}}

Durante buena parte del siglo XX, los puntos de vista sobre las clases tuvieron una fuerte influencia en los debates de las ciencias sociales. Los postulados de Marx fueron cuestionados porque, supuestamente, la importancia relativa de la clase obrera disminuía con el tiempo en tanto que las clases medias ---sobre todo, durante los ``años felices'' del Estado de Bienestar 'a la europea'v se incrementaban y convivían con la considerable expansión de las diferencias basadas en nuevas categorías de análisis tales como género, edad o etnia. Y si bien la acción asume un rol importante en las explicaciones marxistas de la actualidad, por el contrario, hay reticencias a la incorporación de formas de explicación estructural por parte de los weberianos (\protect\hyperlink{ref-Burris1995}{Burris, 1995, p. 136}).

El rápido cambio social, económico y político, a nivel mundial, durante los noventa (dominado por el Consenso de Washington) comenzó a minar una parte de las presunciones y preocupaciones del análisis empírico de las clases sociales, en gran medida como consecuencia de los cambios en la estructura ocupacional y la declinación de la clase obrera tradicional en los países industrializados, con el correlato decrecimiento de ocupaciones no-manuales -que convencionalmente se las ubicó en la clase media-.

El cambio tecnológico, los cambios en la división internacional del trabajo, el impacto de la crisis del petróleo, la disminución en importancia y en número de industrias tradicionalmente reclutadores de clase obrera, el aumento paralelo de la denominada ``economía de servicios''; los mayores niveles del bienestar y la ampliación de las clases medias, la emergencia de nuevas categorías de análisis, como el género y la etnia se han tomado como algunos de los factores de la declinación de empleos industriales.

Este cambio técnico, el mayor dinamismo del sector terciario, la pérdida de liderazgo de la industria de ``modelo fordista'', la caída de salarios, el aumento del empleo femenino y nuevas configuraciones en el mundo del trabajo, fueron algunos de los puntos señalados para mostrar la captación de estas nuevas realidades utilizando las variables tradicionales del análisis de la estructura social. Muchos críticos de la noción de clase cuestionaron que el empleo, debido a su debilitada importancia a fines del siglo XX, pueda ser el punto partida para la identidad social, desde al menos, la segunda mitad del siglo XX.

Previo a los cuestionamientos sobre el cambio el cambio técnico, algunos autores se cuestionaban acerca de la ``muerte'' del concepto de clases sociales. Nisbet (\protect\hyperlink{ref-Nisbet1959}{1959}) y sus continuadores justificaron la eventual desaparición de la noción de clase. Lo fundamentaban en que el ámbito político se distinguía una difusión del poder en todos los segmentos de la población y una ruptura del comportamiento político de acuerdo al estrato social; en el ámbito económico, aumentaba en el sector terciario, cuyos puestos de trabajo no coincidían con la mayoría de cualquier sistema de clase, y por la distribución de la propiedad en todos los estratos sociales y los niveles de vida y consumo que llevaron a la desaparición de los estratos claramente identificables de consumo.

Otra corriente teórica que se oponía al uso del concepto de clase fue la denominada teoría de la ``sociedad post-industrial'' en la que destacan autores tales como Bell y Touraine. Planteaban la convergencia en todas las sociedades hacia un aumento de los requisitos de conocimiento en el lugar de trabajo, menor desigualdad social y desgaste de la burguesía como clase dominante. Bell (\protect\hyperlink{ref-Bell2006}{2006}) tomó en cuenta tres dimensiones analíticas de la sociedad: \textbf{la estructura social}, en la que incluye la economía, la tecnología y el sistema de trabajo; la \textbf{política}, que regula la distribución del poder y la \textbf{cultura}, lugar del simbolismo y los significados.

Muchas de las interpretaciones que cuestionan el concepto de clase sostienen que la clase obrera perdió su centralidad en la teoría y que el desarrollo económico se traduce en una decreciente importancia de la clase social como base de la acción política. Otorgan su lugar a los nuevos movimientos sociales, como la corrientes ``postmaterialismo'' (\protect\hyperlink{ref-Inglehart1990}{Inglehart, 1990}) que sostiene la idea de que a medida que aumenta el bienestar económico de una sociedad, los valores que conforman a sus individuos no son guiados por pautas materialistas.

Clark y Lipset (\protect\hyperlink{ref-Clark1991}{1991}) preguntaban a comienzos de la última década del siglo XX si las clases se estaban ``muriendo''. Esta vieja posición, favorable a la desaparición de las clases, consideraba que la transformación de la estructura ocupacional junto con el desarrollo del Estado Benefactor se tradujo en un mayor nivel de riqueza y bienestar en la población y en nuevas formas de comportamiento político (tales como la ecología o las libertades civiles). La declinación de la dinámica tradicional de izquierda-derecha de los partidos políticos y la emergencia de divisiones sociales no ancladas en las clases, tales como el aumento de las credenciales educativas, nuevas formas en el mercado de trabajo, o las desigualdades geográficas y locales, lograron un declive de la política de clase, que se manifestaba por un lado en una inexistencia de correlación entre voto y clase, la declinación de los partidos políticos tradicionales, la ampliación de las clases medias y el ``reciclaje'' de partidos tradicionales de izquierda.

Hout y otros (\protect\hyperlink{ref-Hout1993}{1993}), siguiendo el debate, argumentaron que el concepto de clase es fundamental en sociología ya que es indispensable para determinar intereses materiales. Las clases definidas ``estructuralmente'' pueden dar lugar a la emergencia de acción colectiva y la pertenencia a una clase incide sobre las oportunidades de vida y diversos aspectos relevantes de la vida social. La propiedad, las diferencias de renta y riqueza siguen estando de forma persistente asociadas a la clase social.

Al respecto, es interesante el razonamiento que propone Filgueira cuando considera que

\begin{quote}
``Si uno de los problemas del paradigma clásico es su excesiva atadura a la dimensión empleo, la consideración de las diversas formas de capital abren el camino para la discusión de otros principios ordenadores de la diferenciación social. En particular, la cuestión actualmente en debate acerca de la relevancia de los estilos de vida y consumo como alternativa estructuradora de la diversidad del orden social en oposición o en paralelo con la determinación del empleo, o si se quiere, acerca de la pregunta sobre la vigencia actual del''paradigma productivista''. Naturalmente, ello conduce a la pregunta acerca de qué ha cambiado para que la necesidad de revisar el paradigma clásico se haga evidente. No parece haber una respuesta definitiva acerca del punto aunque las razones que se aducen se remiten a la incapacidad de la teoría sociológica contemporánea de dar respuestas idóneas a las recientes transformaciones macroestructurales. En principio, tal hipótesis no parece del todo plausible o por lo menos está mal formulada. La carencia explicativa del paradigma clásico existió con independencia de los ``cambios objetivos''. Lo que sí ocurrió es que estos cambios hicieron más evidente sus limitaciones'' (\protect\hyperlink{ref-Filgueira2001}{Filgueira, 2001, p. 22})
\end{quote}

No ha sido la primera vez que el término de ``clase'' y el análisis de clase se viera objetado como desfasado en el tiempo y que, sus estrategias de investigación y abordaje fueran consideradas irrelevantes para sociedad actual. Si bien durante la mayor parte de los siglos XIX y XX el tema de las clases sociales ha estado en el centro de los debates académicos y extra-académicos de los científicos sociales ---tal vez como reflejo de una discusión política más amplia acerca de las luchas sociales en las sociedades industrializadas occidentales (\protect\hyperlink{ref-Bauman2000}{Bauman, 2000}; \protect\hyperlink{ref-Furbanck2005}{Furbanck, 2005})---, cada tanto suele desaparecer de la corriente política o de la agenda mediática. Como otros paradigmas de la ciencia y de la sociología, el tema de las clase sociales parece una gran ``estrella púlsar'' que ilumina y decae a lo largo del tiempo.

\hypertarget{cuestion}{%
\subsection{La cuestión en América Latina y en la Argentina}\label{cuestion}}

¿Cómo estudian los cientistas sociales la desigualdad social en términos de clases sociales, en general, y en particular en la Argentina? ¿Qué fuentes de información están disponibles y cuáles se utilizan para responder preguntas? ¿Qué temas investigan los cientistas sociales relativos a las clases sociales en Argentina? ¿Cuáles son las tendencias históricas de los principales temas del estudio de las clases sociales? ¿Qué relevancia tienen los enfoques propuestos para interpretar la realidad social contemporánea?

En América Latina los debates originados en la sociología de los países centrales fueron seguidos de cerca, y en muchos casos, problematizados y ampliados con propuestas originales y novedosas, que anticiparon los trabajos de Wright o Goldthorpe. Filgueira (\protect\hyperlink{ref-Filgueira2001}{2001}) sostuvo que el paradigma de la estratificación social en América Latina se enraíza en las tradiciones intelectuales europeas, principalmente en la corriente marxista y la weberiana. Lo denominó ``paradigma clásico'' de los estudios de estratificación y movilidad social. En el desarrollo inicial de este tipo de estudios en Latinoamérica se destaca la influencia de Gino Germani (\protect\hyperlink{ref-Germani1987}{1987}), José Medina Echavarría (\protect\hyperlink{ref-MedinaEchavarria1964}{1964}), Florestán Fernándes (\protect\hyperlink{ref-Fernandes1998}{1998}) así como de Solari en Uruguay, Hutchinson en Brasil y Costa Pinto en Chile, entre otros.

Filguiera (\protect\hyperlink{ref-Filgueira2001}{2001}) puso en evidencia una inexplicable postergación en la investigación sobre el tema en el último cuarto del siglo XX. Observaba que había una relativa carencia de estudios de estratificación en la agenda de investigaciones en América Latina y que si bien no se interrumpió, fue perdiendo lugar frente a otras temáticas. Consideró importante retomarlo por la necesidad de lograr un marco comprensivo de los cambios producto de la globalización y el cambio técnico y por las limitaciones conceptuales del tradicional paradigma de la estratificación. De esta manera se dilucidaría el impacto de las transformaciones en el mercado de trabajo y propuso un abandono del ``paradigma productivista'' clásico de la estratificación por una visión renovada del enfoque de clases que integre dimensiones tales como capital social, regímenes de bienestar y cambio demográfico. Sostiene que

\begin{quote}
``Uno de los factores contribuyentes a la pérdida de relevancia de los estudios de estratificación y movilidad social comparados en la región, estuvo dado por el vuelco de los estudios sociales hacia los problemas de pobreza y exclusión social {[}\ldots{]} Como resultado, sobre América Latina conocemos hoy día por ejemplo, muchas más sobre los pobres, los indigentes y los marginales que sobre las condiciones de vida, alineamientos sociales y movilidad de las clases bajas urbanas integradas o de las `clases medias'\,'' (\protect\hyperlink{ref-Filgueira2001}{2001, p. 8}).
\end{quote}

En el trabajo ya clásico de Filgueira y Geneletti (\protect\hyperlink{ref-Filgueira1981}{1981}) se realizaban interesantes observaciones sobre los cambios en la estructura social en base a datos censales. Barozet (\protect\hyperlink{ref-Barozet2007}{2007}) señala que algunos estudios de orientación cepalina como el de Portes (\protect\hyperlink{ref-Portes2003}{2003}) reintrodujeron la noción de clase social desde un punto de vista marxista, con una mirada regional y temporal y muchos otros hicieron foco en las clases medias latinoamericanas como los de Franco y Hopenhayn (\protect\hyperlink{ref-Franco2010}{2010}) y Franco, León y Atria (\protect\hyperlink{ref-Franco2010a}{2010}; \protect\hyperlink{ref-Franco2007}{2007}).

A pesar del fuerte impulso inicial que se dio entre 1950 y 1970 a los estudios sobre estratificación, movilidad y clases sociales en América Latina, fue perdiendo de a poco su ímpetu hasta volverse muy menor en la agenda de investigación en la última década del siglo XX. Como se mencionó anteriormente, alguna de las razones de esta caída pudo deberse a la predominancia del enfoque neoliberal en la economía de la región durante las décadas de 1980 y 1990, cuando la cuestión social comenzó a cobrar relevancia con los problemas de empleo, pobreza y exclusión social.

\hypertarget{nuevos-aires}{%
\subsubsection{Nuevos aires}\label{nuevos-aires}}

Ya a comienzos del siglo XXI las investigaciones sobre clases sociales vuelven a la agenda de muchos institutos. Es que los procesos de cambio que se dieron en la región durante la primera década del dos mil dieron lugar a interrogantes acerca de sus efectos sobre la estructura de clases y la desigualdad social, que el enfoque de la estratificación social podía ayudar a responder.

En sintonía con la expansión de la sociología académica en la Argentina, la ``teoría de clases'' asumió una importancia primordial en los debates sociológicos durante fines de la década del cincuenta y hasta mediados de los '70. La investigación académica sobre estructura social fue un tema frecuente, cuando a partir la ``escuela'' de Germani (\protect\hyperlink{ref-Germani1987}{1987}) fue abordado con mayor sistematización y dio lugar a numerosos debates en la sociología argentina.

Durante la década de 1950 se impulsaron en América Latina una serie de investigaciones considerando a la clase media como el sector clave para la estabilidad social y el desarrollo, que se manifestaba y caracterizaba, entre otras cosas, por movilidad ascendente. Emparentado con una tradición idealista, Poviña (\protect\hyperlink{ref-Povina1950}{1950}) caracterizó en términos culturales y espirituales a la clase media, mientras que Bagú (\protect\hyperlink{ref-Bagu1959}{1959}) lo hizo en términos psicosociales. Germani (\protect\hyperlink{ref-Germani1944}{1944}, \protect\hyperlink{ref-Germani1963}{1963}, \protect\hyperlink{ref-Germani1981}{1981}), propició estudios con sólido sustento empírico y de largo plazo, llevando a cabo investigaciones bajo ``indicadores objetivos de clase'' y de autoidentificación, pensando a su vez las correspondencias entre movilidad y estratificación.

Durante la década del '60 y del '70 del siglo XX, diversos estudios centraron su atención en la relación entre orientación ideológica y acción política de la clase media (\protect\hyperlink{ref-Graciarena1971}{Graciarena, 1971}; \protect\hyperlink{ref-Tedesco1973}{Tedesco, 1973}). Quizás el estudio más importante de la época lo constituyó el de de Ípola y Torrado (\protect\hyperlink{ref-Ipola1976}{1976}), quienes abordaron el análisis de las clases sociales a partir de una interpretación del fenómeno basado en una conceptualización renovada de las clases sociales desde materialismo histórico, trabajo que será retomado luego por Torrado (\protect\hyperlink{ref-Torrado1992}{1992}).

Durante los noventa y luego de la crisis del 2001, el debate acerca de las clases sociales comienza a expandirse, poniéndose ``de moda'' los estudios de acerca de las clases medias, con investigaciones que fueron dando cuenta del deterioro creciente de las condiciones de vida de este sector, producto de la implementación de políticas de ajuste durante la dictadura cívico-militar, extremadas en la última década del siglo XX.

Minujín y Anguita (\protect\hyperlink{ref-Minujin2004}{2004}) analizaron cómo el segmento social distintivo de la Argentina decae en los últimos treinta años del siglo XX. Efectivamente, hasta 1976 la Argentina mostró altos niveles de movilidad ascendente ---intergeneracional e intrageneracional--- corroborados en la jerarquía ocupacional y una movilidad ascendente del nivel de ingreso (\protect\hyperlink{ref-Torrado2007}{Torrado, 2007}). Luego de esa fecha, tanto la movilidad ocupacional como de ingresos mostraron flujos descendentes, siendo la ocupación y el trabajo precario nuevas modalidades de trabajo para este estrato (\protect\hyperlink{ref-Lindenboim2007}{Lindenboim, 2007}; \protect\hyperlink{ref-Monza1986}{Monza \& Planificación, 1986}; \protect\hyperlink{ref-Orsatti1985}{Orsatti, 1985}).

Diversas publicaciones mostraron el proceso de pauperización creciente y los efectos del proceso de concentración que involucró el surgimiento de una nueva categoría, la de los ``nuevos pobres'' o ``pobres por ingresos'' donde se encontraban fracciones de la clase media afectadas por la precarización de las condiciones de trabajo y el desempleo (\protect\hyperlink{ref-Minujin1992}{Minujín, 1992}; \protect\hyperlink{ref-Minujin1995}{Minujín \& Kessler, 1995}; \protect\hyperlink{ref-Murmis1992}{Murmis \& Feldman, 1992}), disminuyendo así, no sólo su peso relativo sino también mudándolo a ser un sector vulnerable a la pobreza (\protect\hyperlink{ref-Altimir1999}{Altimir \& Beccaria, 1999}; \protect\hyperlink{ref-Beccaria2007}{Beccaria, 2007}; \protect\hyperlink{ref-LoVuolo2004c}{Lo Vuolo, Barbeito, Pautassi, \& Rodríguez, 2004}).

El problema de la distribución del ingreso contribuyó largamente al análisis de los cambios observados en la clase media y en las clases sociales en general. Distintos trabajos mostraron las contradicciones y pujas en la estructura social de países como la Argentina -por ejemplo (\protect\hyperlink{ref-Lindenboim2007}{Lindenboim, 2007})- y dieron cuenta cómo este proceso se acentuó, como también lo hacía la desigualdad social, en niveles altamente diferentes con respecto al pasado (\protect\hyperlink{ref-Lopez2005a}{López \& Romeo, 2005}). La expansión de la pobreza, proyectaba una clase media en retroceso cuantitativo, transformando los patrones de su estructura y movilidad, asimilando a la Argentina a otros países de América Latina. En líneas generales, los estudios especializados señalaron cómo aquel conjunto poblacional definido como la clase media presentaba movilidad tanto ascendente como descendente y al mismo tiempo cambios estructurales en convergencia con los culturales.

Desde el punto de vista del ``giro cultural'' que menciona Crompton diversos autores (\protect\hyperlink{ref-Garguin2009}{Garguin, 2009}; \protect\hyperlink{ref-Losada2009}{Losada, 2009}; \protect\hyperlink{ref-Visacovsky2009}{Visacovsky \& Garguin, 2009}; \protect\hyperlink{ref-Zimmermann2000}{Zimmermann, 2000}) abordaron distintas dimensiones acerca de la formación de las clases medias argentinas, debate que hoy en día constituye un campo específico de los estudios de estratificación social en la Argentina. Adamovsky, por ejemplo, reflexionó acerca las condiciones históricas ``subjetivas'' que dieron origen a la clase media Adamovsky (\protect\hyperlink{ref-Adamovsky2009b}{2009b}) y a las clases populares (\protect\hyperlink{ref-Adamovsky2012}{2012}) problematizando la formación histórica de las clases sociales. También fueron observadas a partir de sus consumos culturales o económicos ---por ejemplo la investigación de Wortman ((\protect\hyperlink{ref-Wortman2003}{2003})).

Desde el abordaje estadístico, por un lado, hubo una importante aparición de estudios que analizaron los impactos y las diferenciaciones que produjeron los cambios en el modelo de acumulación sobre la estructura de clases. En este sentido pueden nombrarse los aportes de Benza (\protect\hyperlink{ref-Benza2016}{2016}), Chávez Molina \& Sacco (\protect\hyperlink{ref-ChavezMolina2015}{2015}), Iñigo Carrera, Podestá, \& Cotarelo (\protect\hyperlink{ref-InigoCarrera1999}{1999}), Palomino \& Dalle (\protect\hyperlink{ref-Palomino.Dalle2012}{2012}), Pla, Rodríguez de la Fuente, \& Sacco (\protect\hyperlink{ref-Pla.etal2018}{2018}), Sacco (\protect\hyperlink{ref-Sacco2019}{2019}). Por otro lado, también volvió a retomar importancia el estudio de la movilidad social. En este caso, se destacan las investigaciones de Kessler y Espinoza (\protect\hyperlink{ref-Kessler2007}{2007}), Benza (\protect\hyperlink{ref-Benza2012}{2012}), Rodríguez de la Fuente (\protect\hyperlink{ref-RodriguezdelaFuente2020}{2020}) y las de los equipos del área laboral y de estratificación social del Instituto de Investigaciones Gino Germani (\protect\hyperlink{ref-ChavezMolina2009}{Chávez Molina \& Molina Derteano, 2009}; \protect\hyperlink{ref-ChavezMolina2014}{Chávez Molina \& Sacco, 2014}; \protect\hyperlink{ref-Dalle2016}{Dalle, 2016}; \protect\hyperlink{ref-Dalle.etal2018}{Dalle, Jorrat, \& Riveiro, 2018}; \protect\hyperlink{ref-GomezRojas.Riveiro2014}{Gómez Rojas \& Riveiro, 2014}; \protect\hyperlink{ref-Pla2016}{Pla, 2016}; \protect\hyperlink{ref-Sautu2011}{Sautu, 2011}).

\hypertarget{refs}{}
\begin{CSLReferences}{1}{0}
\leavevmode\vadjust pre{\hypertarget{ref-Adamovsky2009a}{}}%
Adamovsky, E. (2009a). \emph{Argentina, ¿un país de clase media?} Buenos Aires: Canal 7 Argentina.

\leavevmode\vadjust pre{\hypertarget{ref-Adamovsky2009b}{}}%
Adamovsky, E. (2009b). \emph{Historia de la clase media argentina : Apogeo y decadencia de una ilusión, 1919-2003} (1a ed.). Buenos Aires: Planeta.

\leavevmode\vadjust pre{\hypertarget{ref-Adamovsky2012}{}}%
Adamovsky, E. (2012). \emph{Historia de las clases populares en la {Argentina} : Desde 1880 hasta 2003}. Buenos Aires: Editorial Sudamericana.

\leavevmode\vadjust pre{\hypertarget{ref-Althusser1968}{}}%
Althusser, L., \& Balibar, E. (1968). \emph{Lire "le {Capital}"}. Paris,: F. Maspero.

\leavevmode\vadjust pre{\hypertarget{ref-Altimir1999}{}}%
Altimir, O., \& Beccaria, L. (1999). El mercado de trabajo bajo el nuevo régimen económico argentino. \emph{Serie Reformas Económicas}, \emph{68}.

\leavevmode\vadjust pre{\hypertarget{ref-Bagu1959}{}}%
Bagú, S. (1959). \emph{Estratificación y movilidad social en {Argentina}}. Rio de Janeiro,.

\leavevmode\vadjust pre{\hypertarget{ref-Barozet2007}{}}%
Barozet, E. (2007). \emph{La variable ocupación en los estudios de estratificación social}. Documento de trabajo: Fondecyt 1060225.

\leavevmode\vadjust pre{\hypertarget{ref-Bauman2000}{}}%
Bauman, Z. (2000). \emph{Trabajo, consumismo y nuevos pobres}. Barcelona, España: Gedisa editorial.

\leavevmode\vadjust pre{\hypertarget{ref-Beccaria2007}{}}%
Beccaria, L. (2007). Pobreza. In S. Torrado (Ed.), \emph{Población y {Bienestar} en {Argentina} del {Primero} al {Segundo} {Centenario}. {Una} historia social del siglo {XX}}. Buenos Aires: Edhasa.

\leavevmode\vadjust pre{\hypertarget{ref-Bell2006}{}}%
Bell, D. (2006). \emph{El advenimiento de la sociedad post-industrial : Un intento de prognosis social}. Madrid: Alianza.

\leavevmode\vadjust pre{\hypertarget{ref-Benza2012}{}}%
Benza, G. (2012). \emph{Estructura de clases y movilidad intergeneracional en {Buenos} {Aires}: ¿El fin de una sociedad de {``amplias clases medias''}?} (PhD thesis). tesis de doctorado, México, El Colegio de México, Centro de Estudios Sociológicos.

\leavevmode\vadjust pre{\hypertarget{ref-Benza2016}{}}%
Benza, G. (2016). La estructura de clases argentina durante la década 2003-2013. In G. Kessler (Ed.), \emph{La sociedad argentina hoy: Radiografía de una nueva estructura social} (pp. 111--139). Buenos Aires: Siglo Veintiuno Editores.

\leavevmode\vadjust pre{\hypertarget{ref-Bourdieu2012}{}}%
Bourdieu, P. (2012). \emph{La distinción. {Criterio} y bases sociales del gusto}. Buenos Aires: Taurus.

\leavevmode\vadjust pre{\hypertarget{ref-Burris1995}{}}%
Burris, V. (1995). La síntesis neomarxista de {Marx} y {Weber} sobre las clases. In J. Carabaña \& A. de Francisco (Eds.), \emph{Teorías contemporáneas de las clases sociales} (3. ed., pp. 128--156). Madrid: Editorial Pablo Iglesias.

\leavevmode\vadjust pre{\hypertarget{ref-Carabana1997}{}}%
Carabaña, J. (1997). Esquemas y estructuras. \emph{Revista Crítica de Ciências Sociais}, (49), 67--91.

\leavevmode\vadjust pre{\hypertarget{ref-ChavezMolina2009}{}}%
Chávez Molina, E., \& Molina Derteano, P. (2009). \emph{La movilidad socio-ocupacional en la mira. {Un} estudio de caso exploratorio para debatir viejas y nuevas cuestiones}. Facultad de Ciencias Económicas de la Universidad Nacional de Buenos Aires.

\leavevmode\vadjust pre{\hypertarget{ref-ChavezMolina2014}{}}%
Chávez Molina, E., \& Sacco, N. (2014). \emph{Reconfiguraciones en la estructura social: Dos décadas de cambios en los procesos distributivos. {Análisis} del {GBA} según en el clasificador de clases ocupacionales basado en la heterogeneidad estructural 1992-2013} {[}\{CD\}-\{ROM\}{]}. Jujuy, Argentina.

\leavevmode\vadjust pre{\hypertarget{ref-ChavezMolina2015}{}}%
Chávez Molina, E., \& Sacco, N. (2015). Reconfiguraciones en la estructura social: Dos décadas de cambios en los procesos distributivos. {Análisis} del {GBA} según en el clasificador de clases ocupacionales basado en la heterogeneidad estructural 1992-2013. In J. Lindenboim \& A. Salvia (Eds.), \emph{Hora de balance: Proceso de acumulación, mercado de trabajo y bienestar. {Argentina}, 2002-2014} (pp. 287--312). Ciudad Autónoma de Buenos Aires: Eudeba.

\leavevmode\vadjust pre{\hypertarget{ref-Clark1991}{}}%
Clark, T. N., \& Lipset, S. M. (1991). Are social classes dying? \emph{International Sociology}, \emph{6}(4), 397--410.

\leavevmode\vadjust pre{\hypertarget{ref-Crompton2008}{}}%
Crompton, R. (2008). \emph{Class and stratification} (3rd ed.). Cambridge: Polity.

\leavevmode\vadjust pre{\hypertarget{ref-Dahrendorf1979}{}}%
Dahrendorf, R. (1979). \emph{Las clases sociales y su conflicto en la sociedad industrial} (M. Troyano de los Ríos, Trans.). Madrid: Rialp.

\leavevmode\vadjust pre{\hypertarget{ref-Dalle2016}{}}%
Dalle, P. (2016). \emph{Movilidad social desde las clases populares: Un estudio sociológico en el área {Metropolitana} de {Buenos} {Aires} 1960-2013}. Buenos Aires: IIGG-CLACSO.

\leavevmode\vadjust pre{\hypertarget{ref-Dalle.etal2018}{}}%
Dalle, P., Jorrat, J. R., \& Riveiro, M. (2018). Movilidad social intergeneracional. In A. Salvia \& J. I. Piovani (Eds.), \emph{La {Argentina} en el siglo {XXI}. {Cómo} somos, vivimos y convivimos en una sociedad desigual. {Encuesta} {Nacional} sobre la {Estructura} {Social}.} Buenos Aires: Siglo Veintiuno Editores.

\leavevmode\vadjust pre{\hypertarget{ref-Desrosieres2000}{}}%
Desrosières, A., \& Thévenot, L. (2000). \emph{Les catégories socioprofessionnelles}. Paris: Ed. la Découverte.

\leavevmode\vadjust pre{\hypertarget{ref-FeitoAlonso1995}{}}%
Feito Alonso, R. (1995). \emph{Estructura social contemporánea : Las clases sociales en los países industrializados} (1. ed.). Madrid: Siglo XXI de España Editores.

\leavevmode\vadjust pre{\hypertarget{ref-Fernandes1998}{}}%
Fernandes, F., \& Zenteno, R. B. (1998). \emph{Las clases sociales en {América} {Latina}: Problemas de conceptualización : Seminario de {Merida}, {Yuc}., 13 al 18 de diciembre de 1971}. Siglo XXI Ediciones.

\leavevmode\vadjust pre{\hypertarget{ref-Filgueira2001}{}}%
Filgueira, C. H. (2001). \emph{La actualidad de viejas temáticas: Sobre los estudios de clase, estratificación y movilidad social en {América} {Latina}}. CEPAL.

\leavevmode\vadjust pre{\hypertarget{ref-Filgueira1981}{}}%
Filgueira, C. H., \& Geneletti, C. (1981). \emph{Estratificación y movilidad ocupacional en {América} {Latina}}. Santiago de Chile: Naciones Unidas.

\leavevmode\vadjust pre{\hypertarget{ref-Franco2010}{}}%
Franco, R., Hopenhayn, M., \& León, A. (2010). \emph{Las clases medias en {América} {Latina}. {Retrospectiva} y nuevas tendencias}. Siglo veintiuno editores Naciones Unidas.

\leavevmode\vadjust pre{\hypertarget{ref-Franco2010a}{}}%
Franco, R., \& León, A. (2010). Clases medias latinoamericanas: Ayer y hoy. \emph{Estudios Avanzados}, \emph{13}, 59--77.

\leavevmode\vadjust pre{\hypertarget{ref-Franco2007}{}}%
Franco, R., León, A., \& Atria, R. (2007). \emph{Estratificación y movilidad social en {América} {Latina}. {Transformaciones} estructurales de un cuarto de siglo}. Santiago, Chile: CEPAL.

\leavevmode\vadjust pre{\hypertarget{ref-Furbanck2005}{}}%
Furbanck, P. N. A. (2005). \emph{Un placer inconfesable o la idea de clase social}. Buenos Aires, Barcelona, México: Paidós.

\leavevmode\vadjust pre{\hypertarget{ref-Ganzeboom1992}{}}%
Ganzeboom, H. B. G., De Graaf, P. M., Treiman, D. J., \& De Leeuw, J. (1992). A {Standard} {International} {Socio}-{Economic} {Index} of {Occupational} {Status}. \emph{Social Science Research}, \emph{21}(1), 1--56.

\leavevmode\vadjust pre{\hypertarget{ref-Garguin2009}{}}%
Garguin, E. (2009). "{Los} argentinos descendemos de los barcos''. {Articulación} racial de la identidad de clase media en {Argentina} (1920-1960). In S. E. Visacovsky \& E. Garguin (Eds.), \emph{Moralidades, economías e identidades de clase media}.

\leavevmode\vadjust pre{\hypertarget{ref-Germani1944}{}}%
Germani, G. (1944). Sociografía de la clase media en {Buenos} {Aires}: {Las} características culturales de la clase media de {Buenos} {Aires} estudiadas a través de la forma de empleo de las horas libres. \emph{Boletín Del Instituto de Sociología, Facultad de Filosofía y Letras, Universidad de Buenos Aires}.

\leavevmode\vadjust pre{\hypertarget{ref-Germani1963}{}}%
Germani, G. (1963). La movilidad social en la {Argentina}. In \emph{Movilidad social en la sociedad industrial}. Buenos Aires: EUDEBA.

\leavevmode\vadjust pre{\hypertarget{ref-Germani1981}{}}%
Germani, G. (1981). La clase media en la ciudad de {Buenos} {Aires}: {Estudio} preliminar. \emph{Desarrollo Económico, Buenos Aires}, \emph{21}(81).

\leavevmode\vadjust pre{\hypertarget{ref-Germani1987}{}}%
Germani, G. (1987). \emph{Estructura social de la {Argentina}; análisis estadístico}. Ediciones Solar.

\leavevmode\vadjust pre{\hypertarget{ref-Giddens1998}{}}%
Giddens, A. (1998). \emph{La constitución de la sociedad: Bases para la teoría de la estructuración}. Amorrortu Editores España SL.

\leavevmode\vadjust pre{\hypertarget{ref-GomezRojas.Riveiro2014}{}}%
Gómez Rojas, G., \& Riveiro, M. (2014). Hacia una mirada de género en los estudios de movilidad social: Interrogantes teórico-metodológicos. \emph{Boletín Científico Sapiens Research}, \emph{4}(1), 26--31. Retrieved from \url{https://www.srg.com.co/bcsr/index.php/bcsr/article/view/74}

\leavevmode\vadjust pre{\hypertarget{ref-Gouldner2000}{}}%
Gouldner, A. W. (2000). \emph{La crisis de la sociología occidental}. Buenos Aires: Amorrortu.

\leavevmode\vadjust pre{\hypertarget{ref-Graciarena1971}{}}%
Graciarena, J. (1971). Clases medias y movimiento estudiantil. {El} {Reformismo} {Argentino}: 1918-1966. \emph{Revista Mexicana de Sociologia}, \emph{33}(1), 61--100.

\leavevmode\vadjust pre{\hypertarget{ref-Hout1993}{}}%
Hout, M., Brooks, C., \& Manza, J. (1993). The persistence of classes in post-industrial societies. \emph{International Sociology International Sociology}, \emph{8}(3), 259--277.

\leavevmode\vadjust pre{\hypertarget{ref-Inglehart1990}{}}%
Inglehart, R. (1990). \emph{Culture shift in advanced industrial society}. PRINCETON University Press.

\leavevmode\vadjust pre{\hypertarget{ref-InigoCarrera1999}{}}%
Iñigo Carrera, N., Podestá, J., \& Cotarelo, M. C. (1999). Las estructuras económico sociales concretas que constituyen la formación económica de la {Argentina}. \emph{Programa de Investigación Sobre El Movimiento de La Sociedad Argentina (PIMSA)}, \emph{Documento de Trabajo Nº 18}.

\leavevmode\vadjust pre{\hypertarget{ref-Ipola1976}{}}%
Ipola, E. de, \& Torrado, S. (1976). \emph{Teoría y método para el estudio de la estructura de clases sociales ({Con} un análisis concreto: {Chile}, 1970)}. Santiago: Flacso-Proelce.

\leavevmode\vadjust pre{\hypertarget{ref-Kerbo2003}{}}%
Kerbo, H. R. (2003). \emph{Estratificación y desigualdad. {El} conflicto de clases en perspectiva histórica, comparada y global} (7th ed.). Madrid, España: McGraw-Hill/ Interamericana de España, S.A.U.

\leavevmode\vadjust pre{\hypertarget{ref-Kessler2007}{}}%
Kessler, G., \& Espinoza, V. (2007). Movilidad social y trayectorias ocupacionales en {Buenos} {Aires}. {Continuidades}, rupturas y paradojas. In R. Franco, A. León, \& R. Atria (Eds.), \emph{Estratificación y movilidad social en {América} {Latina}: Transformaciones estructurales de un cuarto de siglo} (1. ed., pp. 614 p.). Santiago, Chile: LOM Ediciones, CEPAL, GTZ.

\leavevmode\vadjust pre{\hypertarget{ref-Laclau1987}{}}%
Laclau, E., \& Mouffe, C. (1987). \emph{Hegemonía y estrategia socialista : Hacia una radicalización de la democracia}. Madrid {[}etc.{]}: Siglo XXI.

\leavevmode\vadjust pre{\hypertarget{ref-Levine2006}{}}%
Levine, R. F. (2006). \emph{Social class and stratification: Classic statements and theoretical debates}. Rowman \& Littlefield Publishers.

\leavevmode\vadjust pre{\hypertarget{ref-Lindenboim2007}{}}%
Lindenboim, J. (2007). La fuerza de trabajo en el siglo {XX}. {Viejas} y nuevas discusiones. In S. Torrado (Ed.), \emph{Población y {Bienestar} en {Argentina} del {Primero} al {Segundo} {Centenario}. {Una} historia social del siglo {XX}}. Buenos Aires: Edhasa.

\leavevmode\vadjust pre{\hypertarget{ref-LoVuolo2004c}{}}%
Lo Vuolo, R., Barbeito, A., Pautassi, L., \& Rodríguez, C. (2004). \emph{La pobreza ... De la política contra la pobreza.} Buenos Aires: Ceipp.

\leavevmode\vadjust pre{\hypertarget{ref-Lockwood1962}{}}%
Lockwood, D. (1962). \emph{El trabajador de la clase media un estudio sobre la conciencia de clase}. Madrid: Aguilar.

\leavevmode\vadjust pre{\hypertarget{ref-Lopez2005a}{}}%
López, A., \& Romeo, M. (2005). \emph{La declinación de la clase media argentina : Transformaciones en la estructura social, 1974-2004} (1. ed.). Buenos Aires: Aurelia Rivera.

\leavevmode\vadjust pre{\hypertarget{ref-Losada2009}{}}%
Losada, L. (2009). \emph{Historia de las elites en la {Argentina}: Desde la {Conquista} hasta el surgimiento del peronismo}. Buenos Aires: Sudamericana.

\leavevmode\vadjust pre{\hypertarget{ref-Martinez2005}{}}%
Martínez, R. (2005). \emph{Estructura social y estratificación} (2° ed.; C. editor de la Universidad Pablo Olavide Sevilla, Ed.). Buenos Aires-Madrid: Miño y Dávila.

\leavevmode\vadjust pre{\hypertarget{ref-MedinaEchavarria1964}{}}%
Medina Echavarría, J. (1964). \emph{Consideraciones sociológicas sobre el desarrollo económico de {América} {Latina}}. Buenos Aires: Solar: Hachette.

\leavevmode\vadjust pre{\hypertarget{ref-Minujin1992}{}}%
Minujín, A. (1992). \emph{Cuesto abajo: Los nuevos pobres: Efectos de la crisis en la sociedad argentina} (1a ed.). Buenos Aires: UNICEF : LOSADA.

\leavevmode\vadjust pre{\hypertarget{ref-Minujin2004}{}}%
Minujín, A., \& Anguita, E. (2004). \emph{La clase media: Seducida y abandonada} (1. ed.). Buenos Aires: Edhasa.

\leavevmode\vadjust pre{\hypertarget{ref-Minujin1995}{}}%
Minujín, A., \& Kessler, G. (1995). \emph{La nueva pobreza en la {Argentina}}. Buenos Aires, Argentina: Editorial Planeta.

\leavevmode\vadjust pre{\hypertarget{ref-Monza1986}{}}%
Monza, A., \& Planificación, S. de. (1986). \emph{El terciario argentino y el ajuste del mercado de trabajo urbano (1947-1980)}.

\leavevmode\vadjust pre{\hypertarget{ref-Murmis1992}{}}%
Murmis, M., \& Feldman, S. (1992). Posibilidades y fracasos de las clase medias, según {Germani}. In J. R. Jorrat \& R. Sautu (Eds.), \emph{Después de {Germani}: {Exploraciones} sobre estructura social de la {Argentina}} (pp. 278 p.). Buenos Aires: Paidos.

\leavevmode\vadjust pre{\hypertarget{ref-Nisbet1959}{}}%
Nisbet, R. A. (1959). The decline and fall of social class. \emph{Pacific Sociological Review}, \emph{2}(1), 11--17.

\leavevmode\vadjust pre{\hypertarget{ref-Orsatti1985}{}}%
Orsatti, Á. (1985). El empleo precario en {Buenos} {Aires}, 1974-1983. In \emph{El empleo precario en {Argentina}}. Buenos Aires: CIAT-Ministerio de Trabajo y Seguridad Social.

\leavevmode\vadjust pre{\hypertarget{ref-Ossowski1972}{}}%
Ossowski, S. (1972). \emph{Estructura de clases y conciencia social}. Barcelona: Península.

\leavevmode\vadjust pre{\hypertarget{ref-Palomino.Dalle2012}{}}%
Palomino, H., \& Dalle, P. (2012). El impacto de los cambios ocupacionales en la estructura social de la {Argentina}: 2003-2011. \emph{Revista de Trabajo}, \emph{10}(8), 205--223. Retrieved from \url{http://www.trabajo.gob.ar/downloads/estadisticas/2012n10_revistaDeTrabajo.pdf}

\leavevmode\vadjust pre{\hypertarget{ref-Parkin1968}{}}%
Parkin, F. (1968). \emph{Middle class radicalism; the social bases of the {British} {Campaign} for {Nuclear} {Disarmament}}. New York,: F. A. Praeger.

\leavevmode\vadjust pre{\hypertarget{ref-Pla2016}{}}%
Pla, J. (2016). \emph{Condiciones objetivas y esperanzas subjetivas. {Movilidad} social y marcos de (in) certidumbre. {Un} abordaje multidimensional de las trayectorias de clase. {Argentina} durante la primera década del siglo {XXI}}. Buenos Aires: Editorial Autores de Argentina.

\leavevmode\vadjust pre{\hypertarget{ref-Pla.etal2018}{}}%
Pla, J., Rodríguez de la Fuente, J., \& Sacco, N. (2018). Clases sociales y condiciones de vida en el {Gran} {Buenos} {Aires} (2003-2013). \emph{Revista Colombiana de Sociología}, \emph{41}(2), 189--231. \url{https://doi.org/10.15446/rcs.v41n2.64743}

\leavevmode\vadjust pre{\hypertarget{ref-Portes2003}{}}%
Portes, A. (2003). \emph{Las estructuras de clase en {América} {Latina}: Composición y cambios durante la época neoliberal}. CEPAL, Divisiâon de Desarrollo Social.

\leavevmode\vadjust pre{\hypertarget{ref-Poulantzas2005}{}}%
Poulantzas, N. (2005). \emph{Les clases sociales en el capitalismo actual}. Buenos Aires, Argentina: Siglo veintiuno editores.

\leavevmode\vadjust pre{\hypertarget{ref-Povina1950}{}}%
Poviña, A. (1950). Concepto de la clase media y proyección argentina. In T. R. Crevenna (Ed.), \emph{Materiales para el estudio de la clase media en la {América} {Latina}}. Washington: Unión Panamericana.

\leavevmode\vadjust pre{\hypertarget{ref-Przeworski1977}{}}%
Przeworski, A. (1977). Proletariat into a {Class}: {The} {Process} of {Class} {Formation} from {Karl} {Kautsky}'s {The} {Class} {Struggle} to {Recent} {Controversies}. \emph{Politics \& Society Politics \& Society}, \emph{7}(4), 343--401.

\leavevmode\vadjust pre{\hypertarget{ref-RodriguezdelaFuente2020}{}}%
Rodríguez de la Fuente, J. J. (2020). \emph{Del origen de clase a las condiciones de vida actuales. {Movilidad} social y bienestar material en la {Ciudad} de {Buenos} {Aires} (2004-2015)}. Buenos Aires: Teseo Press. Retrieved from \url{https://www.teseopress.com/origendeclase/}

\leavevmode\vadjust pre{\hypertarget{ref-Rose2007}{}}%
Rose, D., \& Harrison, E. (2007). The {European} socio-economic classification: A new social class schema for comparative european research. \emph{European Societies}, \emph{9}(3), 459--490. \url{https://doi.org/10.1080/14616690701336518}

\leavevmode\vadjust pre{\hypertarget{ref-Sacco2019}{}}%
Sacco, N. (2019). Estructura social de la {Argentina}, 1976-2011. \emph{Trabajo y Sociedad}, (32), 25--51. Retrieved from \url{https://dialnet.unirioja.es/servlet/articulo?codigo=6856110}

\leavevmode\vadjust pre{\hypertarget{ref-Sautu2011}{}}%
Sautu, R. (2011). \emph{El análisis de las clases sociales: Teorías y metodologías} (Sociología, Ed.). Buenos Aires: Luxemburg.

\leavevmode\vadjust pre{\hypertarget{ref-Tedesco1973}{}}%
Tedesco, J. C. (1973). \emph{Clases sociales y educación en la {Argentina}}. Rosario: Ediciones Centro de Estudios.

\leavevmode\vadjust pre{\hypertarget{ref-Thompson1968}{}}%
Thompson, E. P. (1968). \emph{The making of the {English} working class} ({[}New). Harmondsworth: Penguin.

\leavevmode\vadjust pre{\hypertarget{ref-Torrado1992}{}}%
Torrado, S. (1992). \emph{Estructura social de la {Argentina}, 1945-1983} (2a ed.). Buenos Aires, República Argentina: Ediciones de la Flor.

\leavevmode\vadjust pre{\hypertarget{ref-Torrado2007}{}}%
Torrado, S. (2007). Estrategias de desarrollo, estructura social y movilidad. In S. Torrado (Ed.), \emph{Población y bienestar en la {Argentina} del primero al segundo centenario. {Una} historia social del siglo {XX}}: \emph{Vol.} \emph{I}. Buenos Aires: Edhasa.

\leavevmode\vadjust pre{\hypertarget{ref-Treiman1977}{}}%
Treiman, D. J. (1977). \emph{Occupational prestige in comparative perspective}. New York: Academic Press.

\leavevmode\vadjust pre{\hypertarget{ref-Visacovsky2009}{}}%
Visacovsky, S. E., \& Garguin, E. (2009). \emph{Moralidades, economías e identidades de clase media. {Estudios} históricos y etnográficos}. Buenos Aires: Antropofagia.

\leavevmode\vadjust pre{\hypertarget{ref-Weber2002}{}}%
Weber, M. (2002). \emph{Economía y sociedad: Esbozo de sociología comprensiva} (J. Winckelmann \& J. M. Echevarría, Eds.). México: Fondo de Cultura Económica.

\leavevmode\vadjust pre{\hypertarget{ref-Wortman2003}{}}%
Wortman, A., \& Arizaga, C. (2003). \emph{Pensar las clases medias : Consumos culturales y estilos de vida urbanos en la {Argentina} de los noventa} (1. ed.). Ciudad Autónoma de Buenos Aires, Argentina: La Crujía Ediciones.

\leavevmode\vadjust pre{\hypertarget{ref-Wright1995a}{}}%
Wright, E. O. (1995). Reflexionando, una vez más, sobre el concepto de estructura de clases. In J. Carabaña \& A. de Francisco (Eds.), \emph{Teorías contemporáneas de las clases sociales} (3. ed., pp. 263 p.). Madrid: Editorial Pablo Iglesias.

\leavevmode\vadjust pre{\hypertarget{ref-Wright1997}{}}%
Wright, E. O. (1997). \emph{Class counts : Comparative studies in class analysis}. Cambridge ; New York Paris: Cambridge University Press ; Maison des sciences de l'homme.

\leavevmode\vadjust pre{\hypertarget{ref-Zimmermann2000}{}}%
Zimmermann, E. A. (2000). La sociedad entre 1870 y 1914. In \emph{Nueva {Historia} de la {Nación} {Argentina}}: \emph{Vol.} \emph{Tomo 4}. Buenos Aires: Planeta.

\end{CSLReferences}

\end{document}
